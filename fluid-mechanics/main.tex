\documentclass[12pt]{report}
\usepackage{graphicx} % Required for inserting images
\usepackage[a4paper, total={6in, 10in}]{geometry}
\usepackage{datetime, xcolor}
\usepackage{amsthm, thmtools, amsfonts, mathtools, amsmath, xfrac, bm}

\newdateformat{monthyeardate}{\monthname[\THEMONTH] \THEYEAR}

\newtheorem{theorem}{Theorem}[section]
\newtheorem{corollary}{Corollary}[theorem]
\newtheorem{lemma}[theorem]{Lemma}
\newtheorem{axiom}{Axiom}
\newtheorem{proposition}{Proposition}[section]
\newtheorem{definition}{Definition}[section]
\theoremstyle{remark}
\newtheorem*{remark}{Remark}

\title{Fluid Mechanics}
\author{Riddhiman}
\date{\monthyeardate\today}

\let\oldvec = \vec
\renewcommand{\vec}[1]{\oldvec{\mathbf{#1}}}
\newcommand{\uvec}[1]{\boldsymbol{\hat{\textbf{#1}}}}

\begin{document}

\maketitle

\chapter{Fundamental Concepts}%
\label{cha:Fundamental Concepts}

\section{Basic Concepts}%
\label{sec:Basic Concepts}

\section{Continuum Hypothesis}%
\label{sec:Continuum Hypothesis}


\begin{itemize}
    \item \begin{align*}
    \text{Knudsen Number } (\text{kn}) = \frac{\text{Mean free path } (\lambda) }{\text{Chateristic Length} (\text{L})}
\end{align*}
    Continuum hypothesis won't be valid for Kn $ \geq $ 0.01. 
    \item Density $ \rho = \frac{\text{mass}}{\text{volume}} $
    \item Specific volume $ \nu = \frac{1}{\rho} $
    \item Specific weight $ \gamma = \frac{mg}{V} = \rho g $ 
    \item Specific Gravity $ \text{SG} = \frac{\rho}{\rho_{\text{water @ } 227K}} $
\end{itemize}

\section{Forces on a fluid particle}%
\label{sec:Forces on a fluid particle}

\begin{itemize}
    \item Body forces
    \item Surface forces
    \item Normal stress, 
        \begin{align*}
            \sigma_{n} = \lim_{\delta A_{n} \rightarrow 0} \frac{\delta F_{n}}{\delta A_{n}}
        \end{align*}
    \item Shear stress, 
        \begin{align*}
            \tau_{n} = \lim_{\delta A_{n} \rightarrow 0} \frac{\delta F_{t}}{\delta A_{n}}
        \end{align*}
    \item Shear tensor,
       \begin{align*}
          \begin{bmatrix}
              \sigma_{xx} & \tau_{xy} & \tau_{xz} \\ 
              \tau_{yx} & \sigma_{yy} & \tau_{yz} \\ 
              \tau_{zx} & \tau_{zy} & \sigma_{zz} \\ 
          \end{bmatrix}
       \end{align*}
\end{itemize}

\section{Viscosity}%
\label{sec:Viscosity}

\begin{flalign*}
    && \tau_{yx} &= \lim_{\delta A_{y} \rightarrow 0 } \frac{\delta F_{x}}{\delta A_{y}} = \frac{dF_{x}}{dA_{y}} &\\
    \text{And, } &&\text{DR} &= \lim_{\delta t \rightarrow 0} \frac{\delta \alpha}{\delta t} = \frac{d \alpha}{dt} &\\
    \text{Since, } && \delta l = \delta \alpha  \ \delta y &\text{ and } \delta l = \delta u \  \delta t &\\
    \text{we get } && \text{DR} &= \frac{du}{dy}
\end{flalign*}

For Newtonian fluids, 
\[ \tau_{yx} = \mu \frac{du}{dy} \qquad \mu \rightarrow \text{Viscosity} \]

\begin{itemize}
    \item  Unit: $ \frac{\text{N s}}{\text{m}^{2}} $
    \item Kinematic viscosity: $ \nu = \frac{\mu}{\rho} $. Unit of this: $ \frac{\text{m}^{2}}{\text{s}} $ 
    \item For non-Newtonian fluids: $ \tau_{yx} = k \left( \frac{du}{dy} \right)^{n} = k \lvert \frac{du}{dy} \rvert^{n-1} \left( \frac{du}{dy} \right) = \eta \frac{du}{dy} $
    \item $ k \rightarrow \text{consistency index} $, $ n \rightarrow \text{flow behavior index} $, $ \eta \rightarrow \text{apparent velocity} $.  
\end{itemize}

\section{Description and classification of fluid motion}%
\label{sec:Description and classification of fluid motion}

\subsection{Steady vs unsteady}%
\label{sub:Steady vs unsteady}

\begin{equation*}
    \frac{\partial \rho}{\partial t} = 0 \implies \rho = \rho(x, y, z) \qquad
    \frac{\partial \vec{V}}{\partial t} = 0 \implies \vec{V} = \vec{V}(x, y, z) \qquad
    \frac{\partial p}{\partial t} = 0 \implies p = p(x, y, z)
\end{equation*}

\subsection{Uniform vs non-uniform}%
\label{sub:Uniform vs non-uniform}

\begin{equation*}
    \left.\frac{\partial \vec{V}}{\partial x} \right\rvert_{t = \text{const}} = 0
\end{equation*}

where $ x $ is in direction of flow. 

\subsection{Compressible vs incompressible}%
\label{sub:Compressible vs incompressible}

\begin{itemize}
    \item $ \rho = \text{constant} $.
    \item At high pressures, \[ \text{Bulk compressibility modulus } E_{v} = \frac{dp}{\left(\sfrac{d \rho}{\rho}\right)} \]
    \item Water hammer, cavitation
    \item Gases are taken to be incompressible if Mach number (Ma) $ \leq 0.3 $. \[ \text{Mach number } Ma = \frac{\text{flow speed, }V}{\text{local speed of sound, }c} \]
    \item Compressible $ \rightarrow $ Supersonic ($ Ma > 1 $) or subsonic ($ Ma < 1 $).
\end{itemize}

\subsection{Inviscid vs viscid flows}%
\label{sub:Inviscid vs viscid flows}

\begin{itemize}
    \item Inviscid flows $ \rightarrow $ zero viscosity (frictionless flow).
    \item Viscid flows $ \rightarrow $ finite viscosity. 
\end{itemize}

\subsection{Laminar vs turbulent}%
\label{sub:Laminar vs turbulent}

\begin{itemize}
    \item Laminar $ \rightarrow $ smooth layers
    \item Turbulent $ \rightarrow $ fluid particles randomly mix. 
    \item \[ \text{Reynold's Number } Re = \frac{\text{Inertial force}}{\text{Viscous force}} = \frac{\rho \bar{V} L}{\mu} \]
    \item Laminar $ \rightarrow Re \leq 2100 $. Turbulent $ \rightarrow Re \geq 4000 $.
\end{itemize}

\section{Surface tension}%
\label{sec:Surface tension}

\begin{itemize}
    \item Pressure difference in a droplet: 
        \begin{flalign*}
            &&\text{Tensile force due to surface tension }  &= \sigma \pi d &\\
            &&\text{Pressure force on the area }  &= (P_{i} - P_{o}) \frac{\pi d^{2}}{4} &\\
            &&\text{Force balance: } \sigma \pi d &= (P_{i} - P_{o}) \frac{\pi d^{2}}{4} &\\
            \implies && \Aboxed{P_{i} &= P_{o} + \frac{4\sigma}{d}} 
        \end{flalign*}
    \item  $ \text{For a soap bubble: } P_{i} = P_{o} + \frac{8 \sigma}{d} $
    \item Capillary effect force balance: 
        \begin{flalign*}
            &&\text{Weight of fluid} &= (\text{Area} \times \text{height}) \rho g &\\
            &&\text{Verticle component of surface tension} &= \sigma \pi d \cos{\theta} &\\
            \implies && \frac{\pi d^{2} h}{4} \rho g &= \sigma \pi d \cos{\theta} &\\
            \implies && \Aboxed{h &= \frac{4 \sigma \cos{\theta}}{\rho g d}}
        \end{flalign*}
        
\end{itemize}

\chapter{Fluid statics}

\section{Pressure formula}%
\label{sec:Pressure formula}

Imagine a small cube of fluid with sides $ dx, dy, dz $.  Assume this fluid is stationary relative to the stationary rectangular coordinates. Let $ \vec{O} $ denote the center of the cube and the pressure there be $ P( \vec{O} ) $. We want to do a force balance on this cube. There are two types of forces acting on this: Body and surface forces. The body force is just gravity: 

\[ \vec{F}_{B} = \vec{g}\ dm = \rho \vec{g}\ dV = \rho \vec{g}\  dx\ dy\ dz \]

For surface forces we need pressure difference. For pressure across the y direction, let on the face on the left we write the taylor series of $ p $ centered around $ \vec{O} $. Thus, 

\begin{flalign*}
    && p_{L} &= p + \frac{\partial p}{\partial y} \left(y_{L} - y\right) &\\
    \implies && &= p + \frac{\partial p}{\partial y} \left( - \frac{dy}{2} \right) &\\
    \implies && &= p - \frac{\partial p}{\partial y} \frac{dy}{2}
\end{flalign*}

Similarly, 

\begin{flalign*}
    && p_{R} &= p + \frac{\partial p}{\partial y} \left(y_{R} - y\right) &\\
    \implies && &= p + \frac{\partial p}{\partial y}  \frac{dy}{2} 
\end{flalign*}

Thus, force on the left face is $ \left(p_{L}\ dx\ dz \uvec{\j}\right) = \left( p - \frac{\partial p}{\partial y} \frac{dy}{2} \right)\ dx\ dz ( \uvec{\j} ) $ and that on the right face is $ \left( p_{R} dx dz \left(- \uvec{\j}\right) \right) = \left( p + \frac{\partial p}{\partial y} \frac{dy}{2} \right)\ dx\ dz \left( - \uvec{\j} \right) $. Force different in this direction gives us: $ \left( p_{R} \left( - \uvec{\j} \right) - p_{L} \left( \uvec{\j} \right) \right)\ dx\ dz = - \left( \frac{\partial p}{\partial y}\ dx\ dy\ dz \right) $ Doing the same for other directions we get, 

\begin{flalign*}
    && d\vec{F}_{S} &= - \left( \frac{\partial p}{\partial x} \uvec{\i} + \frac{\partial p}{\partial y} \uvec{\j} + \frac{\partial p}{\partial z} \uvec{k} \right)\ dx\ dy\ dz &\\ 
    && &= - \nabla p\ dx\ dy\ dz
\end{flalign*}

Ultimately: 

\begin{flalign*}
    && \vec{F} &= d\vec{F}_{B} + d\vec{F}_{S} &\\
    && &= \left( -\nabla p + \rho \vec{g} \right) dV
\end{flalign*}

For a static fluid, $ \vec{F} = m\vec{a} = 0 $. Thus, we simplify the previous equation. Since $ \vec{g} $ points in the $ - \uvec{\j} $ direction, all terms cancel but
\[ \frac{dp}{dz} = - \rho g \]

\end{document}

